\documentclass[a4paper,12pt]{article}
%\usepackage{fontspec}
%\usepackage{unicode}
%\usepackage{polyglossia}
\usepackage[utf8]{inputenc}			%package langue
\usepackage[T1]{fontenc}				%package de langue
\usepackage[french]{babel}				%package langue
\usepackage{lmodern}					% Pour changer le pack de police
%\usepackage{amsmath}
%\usepackage{amssymb}

%\usepackage{ledmac}				%numeroter ligne
%\usepackage{ledpar}					%numeroter paragraphe
%\usepackage{tikz} 					%arbre "philogénétique"
%\usepackage{layout}					%gabarit
%\usepackage{geometry}				%modification des marges
\usepackage{setspace}				%interligne
%\usepackage{soul}					%barrer du texte
%\usepackage{ulem}					%soulignement
%\usepackage{eurosym}				%symbole euro
%\usepackage{bookman}				%pack de police
%\usepackage{charter}				% pack de police
%\usepackage{newcent}				%pack de police
%\usepackage{mathpazo}				%pack de police
%\usepackage{mathptmx}				%pack de police
%\usepackage{url}					% citation url
%\usepackage{verbatim}				%citation de code
%\usepackage{moreverb}				%citation de code
%\usepackage{listings}				%citation code coloré
%\usepackage{fancyhdr}				%en tete et pied de page
%\pagestyle{fancy}
%\usepackage{color}					%colorer le texte
%\usepackage{colortbl}				%colorer texte dans un tableau
%\usepackage[backend=biber, citestyle=verbose-trad1, bibstyle=verbose]{biblatex}%bibli cool
%\usepackage{amsmath}				%symbole mathématique
%\usepackage{amssymb}				%symbole mathematique
%\usepackage{mathrsfs}				% symbole mathematique
%\usepackage{asmthm}				% symbole mathematique
\usepackage{csquotes}				%guillemets francais
%\usepackage{imakeidx}				%index
%\usepackage[pdftex=true]{hyperref}					%metadonnés
%\usepackage{graphicx}				%image
%\usepackage[final]{pdfpages}			%insérerpage pdf
%\usepackage{eurosym}


\usepackage{glossaries}
\makeglossaries



\title{Ethique de la donnée.}
\date{\today}
\author{C. Bibard, E. Nahuet, J-C. Burlot, K. Goarand, P. Allee}



\begin{document}
\maketitle
\tableofcontents





\section{Introduction (pitch 1)}


\subsection{Ethique, morale, déontologie : }

Il existe une confusion entre ethique morale, déontologie. La morale est liée à la notion de bien/mal. Celle-ci est liée à une échelle de valeurs liés à une culture, une religion. Elle est plutôt de l'ordre de l'individuel. 

La déontologie, quand à elle est liée plutôt à des valeurs concernant un corps de métiers. C'est un ensemble de régles généralement corrélée avec la loi. 

\subsection{Ethique médicale :}

L'éthique est une tentative de faire mieux. La question de l'éthique c'est surtout posée dans le domaine médicale. Celui-ci a la particularité de toucher le domaine de santé et de poser des questions de vie ou de mort et appel bien souvent une réponse rapide. 

Dans cette perspective une discution est née entre Tom Beauchamp et Jame Childress, ceci étant de confession réligieuse différente les discutions on donnée lieux à des débats consigné dans le livre *Les principes de l'éthique biomédicale* qui évolue est en est à sa 5e version (unique traduction française).
Ils créent une éthique principiste (correspondant à un certain nombre de principes), le principe d'autonomie, le principe de bienfaisance, le principe de non malfaisance.

\subsubsection{Autonomie : }

Le principe est le principe fondamental dans la prise de décision médicale. Celui-ci s'attache à entendre la parole du patient et à en prendre compte quand cela est possible. Dans le cas d'une personne incapable d'exprimer sa décision, ou de raisonné (folie, démence), il est généralement possible de prendre une décision basée, sur un supposé souhait de sa part. Dans ce cas il semble interessant d'articuler se principe et bouger les curseurs en fonction des autres principes. 


\subsubsection{Bienfaisance : }

Cette idée de bienfaisance est de tenter de faire bien. 


\subsubsection{Non malfaisance : }

L'idée de non malfaisance est, elle en lien, avec le fait de ne pas faire mal. Il est dans les soins médicaux d'avoir à faire des examens, ou à utiliser des traitements invasif, qui mettent à mal le corps (chimio thérapie, par exemple). Ce principe est directement en lien avec la bienfaisance. Il est parfois difficile d'articuler les deux, et il s'agit alors de faire un arbitrage sur un potentiel mieux liés au mal subit. 


\subsubsection{Justice / Equité : }

Ce topic conserne le fait de ne pas faire de discrimination entre les individus. Il est liés au fait qu'un individu est à l'interieur d'une communauté potentiellement inégalitaire, mais que dans la prise en charge celui-ci doit être traité avec ses spécifitié. 

\subsection{La donnée : }
Une donnée est une description élémentaire d’une réalité. C’est par exemple une observation ou une mesure.

La donnée est dépourvue de tout raisonnement, supposition, constatation, probabilité. Étant indiscutable ou indiscutée, elle sert de base à une recherche, à un examen quelconque.

Les données sont généralement le résultat d'un travail préalable sur les données brutes qui permettra de leur donner un sens et ainsi, d'obtenir une information. Les données sont un ensemble de valeurs mesurables en fonction d'un étalon de référence. La référence utilisée et la manière de traiter les données (brutes) sont autant d’interprétations implicites qui peuvent biaiser l’interprétation finale (limites des sondages).

Par exemple, des données dans un graphique permettront à un être humain d'y associer un sens (une interprétation) et ainsi créer une nouvelle information.

\section{Transposition vers l'éthique de la donnée : }
Nous ne proposerons pas ici, une tranposition point à point de ces principes. C'est principe ne sont ni plus ni moins qu'une méthodologie de traitement éthique de ces questions. 
Nous proposons ici, plus proche du monde numérique, une éthique hacker. L'éthique hacker dans un premier temps décrite et introduite par Steven LEVY puis théorisé plus largement par Peka HIMANEN. Considérons, notre présence dans ce hackathon de mettre en pratique nos intuitions sur ce qu'est le hack, nous proposons pour l'éthique de la donnée de reprendre :
	\begin{description}
		\item [L'opensource : ] pour qu'un logiciel soit considéré open source il faut qu'il remplice dix critères :
		\begin{itemize}
			\item  La redistribution doit être libre
			\item Le programme doit être distribué avec le code source, sinon il doit y avoir un moyen très médiatisé pour l’obtenir sans frais
			\item La licence doit autoriser les modifications et les œuvres dérivées, et doit leur permettre d'être distribuées sous les mêmes termes que la licence du logiciel original
			\item Pour maintenir l’intégrité du code source de l'auteur, la licence peut exiger que les œuvres dérivées portent un nom ou un numéro de version différent de ceux du logiciel original
			\item La licence ne doit discriminer aucune personne ou groupe de personnes
			\item La licence ne doit pas défendre d'utiliser le programme dans un domaine d'activité spécifique
			\item Les droits attachés au programme doivent s'appliquer à tous ceux à qui il est redistribué, sans obligation pour ces parties d'obtenir une licence supplémentaire
			\item La licence ne doit pas être spécifique à un produit
			\item La licence ne doit pas imposer des restrictions sur d'autres logiciels distribués avec le logiciel sous licence. Par exemple, la licence ne doit pas exiger que tous les autres programmes distribués sur le même support doivent être des logiciels open source 
			\item La licence doit être technologiquement neutre
		\end{itemize}	
Selon Richard Stallman, l'un des premiers penseur du logiciel libre : \textit{l'open source est une méthodologie de développement; le logiciel libre est un mouvement social}
% http://www.developpez.com/actu/87401/Logiciel-libre-et-open-source-les-deux-concepts-sont-parfois-utilises-de-maniere-interchangeable-mais-quelle-est-la-difference/
		\item [La sécurisation des données : ] Les paquets transmit sur le réseau doivent être signés et chiffrés de bout en bout. Cela afin de garantir la confidentialité et l'intégrité de l'information transmise. 
		\item [Permissionless : ]
		\item [L'échange de pair à pairs : ]
	\end{description}

\section{Etude de cas : Le cas Volswagen (pitch 2)}
Le 20 septembre dernier, le scandal Volkswagen, éclate. Les capteurs posées sur la voiture pour mesurer le taux d'émission de CO2 a été réglé de manière à donner des valeurs inférieures à celles émisent réelement. En dehors d'une véritable fraude, c'est aussi un cas pratique de l'intêret de l'opensource dans ce genre de système, système prévu pour participer à l'effort sur la diminution de la polution, ayant un véritable intêret collectif. 

Peu après la sortie de ce scandal, l'Electronic Frontier Foundation\footnote{Fondation américaine qui défend les droits des citoyens dans le monde numérique}, prend position sur ce scandal en affirmant qu'avec un logiciel Open Source, il n'y aurait pas eu besoin, de la prise de risque d'un lanceur d'alerte, et que la communauté ayant possibilité d'auditer le code source aurait pu faire remonter la triche.

D'autre part, il est probable que l'obligation de rendre public les codes sources, aurait empêché l'entreprise de tricher, l'obligation d'ouverture du code aurait, dans ce cas là, été vertueux. 
\newpage
\section*{Glossaire}

\end{document}










